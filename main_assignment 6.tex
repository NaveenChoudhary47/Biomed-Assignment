\documentclass[12pt]{article}
\usepackage[english]{babel}
\usepackage{natbib}
\usepackage{url}
\usepackage[utf8x]{inputenc}
\usepackage{amsmath}
\usepackage{graphicx}
\graphicspath{{images/}}
\usepackage{parskip}
\usepackage{fancyhdr}
\usepackage{vmargin}
\setmarginsrb{3 cm}{2.5 cm}{3 cm}{2.5 cm}{1 cm}{1.5 cm}{1 cm}{1.5 cm}

\title{5 Solutions to Covid19 provided by Biomedical Engineers}								% Title
\author{21111034}								% Author
\date{25 Jan 2022}											% Date

\makeatletter
\let\thetitle\@title
\let\theauthor\@author
\let\thedate\@date
\makeatother

\pagestyle{fancy}
\fancyhf{}
\rhead{\theauthor}
\lhead{\thetitle}
\cfoot{\thepage}

\begin{document}

%%%%%%%%%%%%%%%%%%%%%%%%%%%%%%%%%%%%%%%%%%%%%%%%%%%%%%%%%%%%%%%%%%%%%%%%%%%%%%%%%%%%%%%%%

\begin{titlepage}
	\centering
    \vspace*{0.5 cm}
    \includegraphics[scale = 0.20]{logo.jpeg}\\[1.0 cm]	% University Logo
    \textsc{\LARGE  National Institute of Technology\newline\newline Raipur}\\[2.0 cm]	% University Name
	\textsc{\Large assignment 06}\\[0.5 cm]				% Course Code
	\rule{\linewidth}{0.2 mm} \\[0.4 cm]
	{ \huge \bfseries \thetitle}\\
	\rule{\linewidth}{0.2 mm} \\[1.0 cm]
	
	\begin{minipage}{0.4\textwidth}
		\begin{flushleft} \large
			\emph{Submitted To:}\\
			Saurabh Gupta\\
            Asst. Professor\\
            Department of Biomedical Engineering\\
			\end{flushleft}
			\end{minipage}~
			\begin{minipage}{0.4\textwidth}
            
			\begin{flushright} \large
			\emph{Submitted By :} \\
			Naveen Choudhary\\
            21111034\\
        First Semester\\
        Biomedical Engineering\\
		\end{flushright}
        
	\end{minipage}\\[2 cm]
	

\end{titlepage}

\newpage

COVID-19 is one of the most severe global health crises that humanity has ever faced. Engineers like to solve problems. For many, the bigger the better. So it comes as no surprise that biomedical engineers have enlisted in the fight against COVID-19. From biomedical research to manufacturing collaborations, biomedical engineers have applied their ingenuity, expertise and community spirit to answer the call. Within a very short period of time, BME research applied to COVID-19 diagnosis has advanced with ever-increasing knowledge and inventions, especially in adapting available virus detection technologies into clinical practice and exploiting the power of interdisciplinary research to design novel diagnostic tools or improve the detection efficiency. The use of medical devices in the COVID pandemic is the unfortunate indication that the patients are displaying severe respiratory distress symptoms and need a form of assistance to breathe. This section provides an overview of how biomedical engineering is contributing to the management of the COVID-19 pandemic.
 
 \section{Patient Monitoring}
 
 An essential element of the ICU equipment is the monitoring equipment that keeps track of some of the patient vitals especially when they are ventilated and sedated but also during their recovery phase to ensure the regime of ventilation is optimised for their condition. Ventilators already provide their set of patient parameters, but usually patient monitors are separate devices as they continue to be useful after the patient can resume breathing on their own unassisted. One of the key parameters for COVID-19 patient is the amount of oxygen in their bloodstream (SpO2), measured by pulse oximetry which uses optics within a finger clamp. Pulse oximetry tends to be used for the duration of the patient’s stay in ICU. Modern patient monitors provide many more patient parameters all the way to breathing waveforms to enable clinicians to fine tune their care of the patients.
 
 \section{Ventilators}
 
 Patients who cannot breathe spontaneously need to be put on a ventilator. Ventilators are capable of replacing the breath function and patients in an advanced state of respiratory distress are usually intubated and sedated at the beginning of the treatment. Ventilators are capable of replacing the breath function and patients in an advanced state of respiratory distress are usually intubated and sedated at the beginning of the treatment. They are complex systems providing the healthcare professionals with a lot of flexibility to adapt the assisted breathing settings and to be able to wean recovering patients off the ventilator gradually. Modern ventilators are typically closed loop pressure controlled and capable of detecting spontaneous breathing to synchronise assistance for recovering patients. They also enable the control of the composition of the gas the patient breathes from normal air to 100 percent oxygen, usually taking their supply from the hospital’s gas supply network but can also be coupled to oxygen tanks or oxygen concentrators if used in a setting where there is no gas network.
 
 \section{PPE kits}
 
 The COVID-19 pandemic has evidenced the fragility of society and the need for effective and practical ways to protect it. For the general public, the use of face masks as personal protection equipment (PPE) remain the most practical line of defence against SARS-CoV-2 as well as other respiratory viral infections. However, for the wide range of multidisciplinary health care workers more protection is required, as surgical or respirator masks, and these are not intended to be worn for so long as is required in an NHS shift. There is an environmental cost to these disposable items, they do not fit all face shapes, the mask-face seal can be broken while talking, and they apply pressure to the sensitive face skin which can cause discomfort and tissue injury. From the patients' and carers' perspectives, they also obscure the face, which disadvantages people with hearing impairments who rely on lip reading - as well as a human face being reassuring. Therefore, there is great need to develop new practical PPE technologies that can protect the population while ensuring a functioning society.  Several groups of engineers have been developing enhanced PPE technologies including powered air purifying respirators (PAPRs), similar to the commercially available devices were in short supply or removed from sale at the start of the pandemic.
 
 \section{CPAP}
 
 The next step up in treating COVID-19 patients is Continuous Positive Airway Pressure (CPAP) which is initially intended to prevent airways collapse in sleep apnoea patients, but has been shown to be beneficial to COVID patients if applied early enough in the progression of the disease. A well-fitted face mask is an essential component of a CPAP system and as such it is quite intrusive. CPAP is only appropriate for patients who are capable of some breathing strength as CPAP effectively opposes some resistance to expiration. Variants exist that either automatically adjust the level of pressure to the patients breathing characteristics (APAP) or have different levels of pressure for inspiration and expiration (BiPAP). CPAP usually supplies (filtered) air to the patient but most masks have a port for supplementing the supply with oxygen.
 
 \section{Innovating in Pandemic}
 
 Beyond the Ventilator Challenge mentioned above, the pandemic inspired engineers around the country to many innovations. This section list only a few of the innovations the authors are aware if and doesn’t mean to single them out from all the great work which is taking place. Biomedical engineers at DNA Nudge developed the COVID Nudge< test from scratch during the pandemic. A team at Imperial College in London developed JAMVENT, a low-cost emergency ventilator, developed in response to the COVID-19 outbreak. Its design is based on simple pneumatic components, but it is able to perform all the tasks required of an ICU ventilator for COVID-19 patients.
 
 \indent
 
 The Pandemic has not ended yet, there are still rise and falls of the cases across the world. Researchers and Biomedical engineers are still still trying to figure out a solution for the problem. So a lot of innovation can be expected in the field of biomedical engineering.

\end{document}
