\documentclass[12pt]{article}
\usepackage[english]{babel}
\usepackage{natbib}
\usepackage{url}
\usepackage[utf8x]{inputenc}
\usepackage{amsmath}
\usepackage{graphicx}
\graphicspath{{images/}}
\usepackage{parskip}
\usepackage{fancyhdr}
\usepackage{vmargin}
\setmarginsrb{3 cm}{2.5 cm}{3 cm}{2.5 cm}{1 cm}{1.5 cm}{1 cm}{1.5 cm}

\title{Future of Healthcare}								% Title
\author{21111034}								% Author
\date{25 Jan 2022}											% Date

\makeatletter
\let\thetitle\@title
\let\theauthor\@author
\let\thedate\@date
\makeatother

\pagestyle{fancy}
\fancyhf{}
\rhead{\theauthor}
\lhead{\thetitle}
\cfoot{\thepage}

\begin{document}

%%%%%%%%%%%%%%%%%%%%%%%%%%%%%%%%%%%%%%%%%%%%%%%%%%%%%%%%%%%%%%%%%%%%%%%%%%%%%%%%%%%%%%%%%

\begin{titlepage}
	\centering
    \vspace*{0.5 cm}
    \includegraphics[scale = 0.20]{logo.jpeg}\\[1.0 cm]	% University Logo
    \textsc{\LARGE  National Institute of Technology\newline\newline Raipur}\\[2.0 cm]	% University Name
	\textsc{\Large assignment 03}\\[0.5 cm]				% Course Code
	\rule{\linewidth}{0.2 mm} \\[0.4 cm]
	{ \huge \bfseries \thetitle}\\
	\rule{\linewidth}{0.2 mm} \\[1.5 cm]
	
	\begin{minipage}{0.4\textwidth}
		\begin{flushleft} \large
			\emph{Submitted To:}\\
			Saurabh Gupta\\
            Asst. Professor\\
            Department of Biomedical Engineering\\
			\end{flushleft}
			\end{minipage}~
			\begin{minipage}{0.4\textwidth}
            
			\begin{flushright} \large
			\emph{Submitted By :} \\
			Naveen Choudhary\\
            21111034\\
        First Semester\\
        Biomedical Engineering\\
		\end{flushright}
        
	\end{minipage}\\[2 cm]
	
\end{titlepage}

By 2040, health care as we know it today will no longer exist. There will be a fundamental shift from “health care” to “health.” And while disease will never be completely eliminated, through science, data, and technology, we will be able to identify it earlier, intervene proactively, and better understand its progression to help consumers more effectively and actively sustain their well-being. The future will be focused on wellness and managed by companies that assume new roles to drive value in the transformed health ecosystem.Driven by greater data connectivity; interoperable and open, secure platforms; and increasing consumer engagement, 10 archetypes are likely to emerge and will replace and redefine today’s traditional life sciences and health care roles to power the future of health. 

\indent

After years of focusing primarily on digitization for electronic health records and other fundamental capabilities, the health care industry is positioning data for a more consumer-focused role. Newly empowered by cloud technology and other advanced tools, health care organizations increasingly have access to the scale and agility they require to curate exploding volumes of data, from sources as diverse as wearable and internet of things-enabled devices, to create a more connected health care system. Even more important, they can now use that data to deliver advanced analytical insights and drive greater innovation, including improved service and better experiences for patients.Today, integrating digitized information into a revamped data platform can enable dynamic, real-time insights, interoperable data lakes, and advanced, artificially intelligent analytics that can help health care organizations make large, innovative leaps forward, bringing greater value to patients.

\indent

There have been numerous mental health initiatives proposed and implemented at health care organizations and educational institutions. These include mindfulness training, hiring scribes to reduce administrative workload, and offering rewards like prepared meals and housecleaning for work outside of one's clinical requirements. While some of these initiatives have shown promise within the conditions they were implemented in, there is no evidence that such a program will have the same level of success in another location or workforce population. For an initiative to be adopted and successful, it must be based on the needs, desires, and unique situation of the stakeholders who will be affected. Using a strategic, collaborative process to identify problems and develop solutions helps ensure that responses to work stress are relevant, effective, and sustainable. A systems approach to design thinking is one way of creating individualized solutions for individuals, work units, or organizations. By continuously testing and adapting the interventions to the unique situation, a strategic approach to building the well-being of workers and the resilience of organizations can be developed.There are a number of well-being initiatives that have been pursued in recent years, designed to address the widespread issue of burnout and professional dissatisfaction among health professionals. These initiatives have been implemented by both academic institutions and health care organizations, and have been designed to improve the well-being of individual students and health professionals as well as to improve organizational resilience.

\indent

Foetal scanning and surgery is already in place, however these are set to improve. This means that health problems in foetuses will be detected before birth and a plan of treatment can be put into place that the parents can follow for a healthier life for the child.The Internet of Things and wearable technology will spot signs of illness quicker, as our health will be constantly monitored. Devices will recognise changes in your health by comparing it to your data when you were in full health, then alert you before anything bad has a chance to develop further.You will no longer need to tell your doctors your symptoms, you can simply step into a scanner like at the airport, or it could even be as easy as breathing into something. The technology will be able to detect illness, giving the doctor one less job and more time to focus on more important tasks.The scanner will use your data, including lifestyle and diet, and data from the in-depth check up to create a full report on your current health. This will be particularly useful with monitoring chronic diseases such as diabetes and dementia.If an illness has been detected, the doctor will be alerted to come and give the patient a consultation, or the scanner will even be able to dispense the medication needed, just like a vending machine! This allows the patient to get medicine quicker than ever, revolutionising the healthcare system.The medicine that will be dispensed will likely be made by artificial intelligence, meaning the scanner will look through vast amounts of chemical or medical data and use this information to create the drug needed. The drug will be printed out in 3D and placed by the patient’s bedside table.

\indent

Hospitals are likely to have a dramatic change, robots will be used to transport patients around the hospital, and this will also create a more sterile environment for the patients. Diagnosis will be more accurate, as using algorithms and AI will be much better than a diagnosis from a human.One of the risks with robots being present in hospitals, is the fact that it’ll feel very impersonal.The robots will do the boring parts, such as diagnosis but the consultation and advice after will be done by real people. The robots will give the doctors more time to offer a higher quality of service, they can take their time rather than rushing the important things.

 
 



\end{document}
