\documentclass[12pt]{article}
\usepackage[english]{babel}
\usepackage{natbib}
\usepackage{url}
\usepackage[utf8x]{inputenc}
\usepackage{amsmath}
\usepackage{graphicx}
\graphicspath{{images/}}
\usepackage{parskip}
\usepackage{fancyhdr}
\usepackage{vmargin}
\setmarginsrb{3 cm}{2.5 cm}{3 cm}{2.5 cm}{1 cm}{1.5 cm}{1 cm}{1.5 cm}

\title{Emerging Technologies in Healthcare}								% Title
\author{21111034}								% Author
\date{25 Jan 2022}											% Date

\makeatletter
\let\thetitle\@title
\let\theauthor\@author
\let\thedate\@date
\makeatother

\pagestyle{fancy}
\fancyhf{}
\rhead{\theauthor}
\lhead{\thetitle}
\cfoot{\thepage}

\begin{document}

%%%%%%%%%%%%%%%%%%%%%%%%%%%%%%%%%%%%%%%%%%%%%%%%%%%%%%%%%%%%%%%%%%%%%%%%%%%%%%%%%%%%%%%%%

\begin{titlepage}
	\centering
    \vspace*{0.5 cm}
    \includegraphics[scale = 0.20]{logo.jpeg}\\[1.0 cm]	% University Logo
    \textsc{\LARGE  National Institute of Technology\newline\newline Raipur}\\[2.0 cm]	% University Name
	\textsc{\Large assignment 05}\\[0.5 cm]				% Course Code
	\rule{\linewidth}{0.2 mm} \\[0.4 cm]
	{ \huge \bfseries \thetitle}\\
	\rule{\linewidth}{0.2 mm} \\[1.0 cm]
	
	\begin{minipage}{0.4\textwidth}
		\begin{flushleft} \large
			\emph{Submitted To:}\\
			Saurabh Gupta\\
            Asst. Professor\\
            Department of Biomedical Engineering\\
			\end{flushleft}
			\end{minipage}~
			\begin{minipage}{0.4\textwidth}
            
			\begin{flushright} \large
			\emph{Submitted By :} \\
			Naveen Choudhary\\
            21111034\\
        First Semester\\
        Biomedical Engineering\\
		\end{flushright}
        
	\end{minipage}\\[2 cm]
	

\end{titlepage}

\newpage

Healthcare technology is poised to reshape the industry as we know it and provide more advanced and efficient patient care. Just as other industries have had to adapt and evolve as new technologies have emerged, healthcare organizations must keep up with healthcare tech trends not only to stay competitive but to also be able to provide the best possible patient outcomes.

\indent

Emerging technologies in the healthcare industry are being introduced at a rapid pace, bringing with them the promise of improved treatment options and more efficient care. This is especially important for healthcare facilities that are seeking solutions to deal with staffing shortages or other limitations. Some of the healthcare technology advances to watch going forward include:

\indent

Artificial Intelligence is transforming the way healthcare organizations manage and draw insights from the incredible amount of scientific data and patient information that’s available. AI can be used to create and customize treatment plans and medication options for patients in a much faster and precise way than human healthcare teams can do on their own. AI can also help in other ways, such as advancing the field of genomic medicine by analyzing complex genetic information to determine the best course of care for individuals based on their DNA. The hope is that AI can one day improve diagnostic accuracy and even predict health outcomes.

\indent

Emerging health information technology has made it possible to maintain health records in a centralized, cloud-based portal, which provides health care professionals and patients with instant access to medical histories. As such, healthcare providers have all the information they need at their fingertips, which can be crucial in the case of an emergency, if there is a language barrier, or if a patient is unable to communicate. This type of healthcare tech is also ideal for when doctors from different hospitals or medical offices need to collaborate about patients who have complex medical files or diagnoses to determine the most optimal way to treat their condition.

\indent

Today’s smartwatches and other wearables do a lot more than count steps. They can monitor heart rates, track sleep patterns, detect heart issues like atrial fibrillation, take your temperature, act as ECG and blood pressure monitors, and more. Wearing these devices allows patients to monitor their own health, which can help identify potential problems. And, they can also share the reporting and data with their physicians as needed. Wearables can also be helpful to monitor post-surgical patients and track their vital signs. Beyond smartwatches, other wearable medical devices are coming to market (or are already being used) that let patients and their healthcare providers monitor glucose levels, oxygenation levels, and measure hand movement in Parkinson’s patients. In the future, other wearable technology may be embedded in eyeglasses, clothing, and other devices.

\indent

The invention of 3D printing is another new technology in the healthcare industry that is proving to be transformative. This new field of 3D Bioprinting enables physicians to print artificial limbs, organs, joint replacement parts, and bio tissues. In addition, in the field of pharmacology, there are ongoing experiments for printing pills and other medications. Lastly, 3D printers can also help create medical devices and surgical tools.

\indent

The field of robotics has made great strides as well, making it a top healthcare tech trend. Medical robots can help surgeons perform very precise and targeted procedures and therapies. Though the doctors still control the surgery, robots take away the possibility of human errors and can potentially reduce infections. Healthcare robots are also poised to take over clerical and routine tasks to free up nursing and other healthcare professionals to focus more on direct patient care.

\indent

As changes in technology can affect lives and even a single mistake can cause disastrous consequences, health care organizations tend to be conservative. New technology can also be complex or expensive, and some people might not have easy access to the technology causing a slow rate of adoption. In addition, many physicians and health care professionals view technology as too impersonal and time consuming, with some find that paper records are quicker and easier to maintain than entering data into a digital system.However, the current transition to a “pay-for-value” payment system could help health care organizations understand the financial value of implementing new technology.

\indent

The future of medicine and patient care will increasingly rely on health technology, which is why healthcare organizations must embrace emerging healthcare technologies to stay relevant in the coming years. By exploring healthcare tech trends and becoming an early adopter of new innovations, healthcare providers can provide cutting-edge care.


\end{document}
