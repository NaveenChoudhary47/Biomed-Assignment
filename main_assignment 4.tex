\documentclass[12pt]{article}
\usepackage[english]{babel}
\usepackage{natbib}
\usepackage{url}
\usepackage[utf8x]{inputenc}
\usepackage{amsmath}
\usepackage{graphicx}
\graphicspath{{images/}}
\usepackage{parskip}
\usepackage{fancyhdr}
\usepackage{vmargin}
\setmarginsrb{3 cm}{2.5 cm}{3 cm}{2.5 cm}{1 cm}{1.5 cm}{1 cm}{1.5 cm}

\title{Disruptive Innovations in Healthcare}								% Title
\author{21111034}								% Author
\date{25 Jan 2022}											% Date

\makeatletter
\let\thetitle\@title
\let\theauthor\@author
\let\thedate\@date
\makeatother

\pagestyle{fancy}
\fancyhf{}
\rhead{\theauthor}
\lhead{\thetitle}
\cfoot{\thepage}

\begin{document}

%%%%%%%%%%%%%%%%%%%%%%%%%%%%%%%%%%%%%%%%%%%%%%%%%%%%%%%%%%%%%%%%%%%%%%%%%%%%%%%%%%%%%%%%%

\begin{titlepage}
	\centering
    \vspace*{0.5 cm}
    \includegraphics[scale = 0.20]{logo.jpeg}\\[1.0 cm]	% University Logo
    \textsc{\LARGE  National Institute of Technology\newline\newline Raipur}\\[2.0 cm]	% University Name
	\textsc{\Large assignment 04}\\[0.5 cm]				% Course Code
	\rule{\linewidth}{0.2 mm} \\[0.4 cm]
	{ \huge \bfseries \thetitle}\\
	\rule{\linewidth}{0.2 mm} \\[1.0 cm]
	
	\begin{minipage}{0.4\textwidth}
		\begin{flushleft} \large
			\emph{Submitted To:}\\
			Saurabh Gupta\\
            Asst. Professor\\
            Department of Biomedical Engineering\\
			\end{flushleft}
			\end{minipage}~
			\begin{minipage}{0.4\textwidth}
            
			\begin{flushright} \large
			\emph{Submitted By :} \\
			Naveen Choudhary\\
            21111034\\
        First Semester\\
        Biomedical Engineering\\
		\end{flushright}
        
	\end{minipage}\\[2 cm]
	

\end{titlepage}

\newpage

The global pandemic threatened to overwhelm healthcare providers. Healthcare innovators, however, thrived as they looked for ways to continue providing quality care and keep staff and patients safe, while maintaining critical revenue streams.Once seen as a disruptive innovation in healthcare, all four modalities of telehealth all played an outsized role in keeping quality care available. In the early days of the pandemic, healthcare providers offered telehealth for patients by any means necessary — some simply falling back to standalone Zoom meetings to provide video visits until they could stand up more integrated solutions. It worked.

\indent

According to a recent Applause survey of more than 5,000 people, almost half (46 percent) used telehealth services at least once, 89 percent of whom said they had done so in the past year. The majority of patients who used telehealth agreed (56 percent) that they found services easy to use; a smaller group strongly agreed (27 percent). In all, 77 percent of patients stated they enjoyed using telehealth, which largely provided a win-win for patients and providers despite a truncated path to market.Just like that, even the most conservative of executives knew that disruptive innovation in healthcare could become a competitive differentiator. The next decade will see current innovations mature and new disruptive ideas for innovation in healthcare that will deliver even better value, care and experiences for patients.

\indent

When we talk about “disruptive innovation in healthcare,” there are a few ways to define what that means. Healthcare providers have long championed technological innovation as a way to deliver better care, from magnetic resonance imaging to new surgical procedures.But, the patient experience was an area more traditionally neglected in healthcare. Telehealth and the rise of consumerism changed the narrative — today, providers succeed when they provide high-quality patient experiences.The last two years required digital healthcare to scale — the next step for providers is to dig deeper into the patient experience. The modern generation of patients shops for healthcare in a vastly different way from those in the past. Healthcare is now a consumer-centric market, as patients seek care at the quality and price standards that work for them. Changing regulations make it easier for patients to price shop, understand their in- or out-of-network options, evaluate a care provider’s reputation, and seek alternatives. On the provider side, there is still some work to do — namely developing a more integrated experience that helps all aspects of the clinic run smoothly.Disruptive innovation in healthcare will shape both patient care and the patient experience, as healthcare providers try to keep doctors’ schedules booked and hospitals running efficiently at full capacity. The value of digital healthcare is now firmly planted in the brains of executives.

\indent

The healthcare industry is ripe for new patient-empowering technologies. Where, when and what technologies will emerge remains to be seen, but there are several key areas where we can expect disruptive innovation in healthcare.Internet of Things. There are nearly endless use cases for IoT in any industry. Healthcare is no exception, both in terms of patient care and patient experience.Remote patient monitoring (RPM) is one area of growing need as healthcare providers evolve and the point of care expands. Any time a healthcare provider can monitor a patient’s health without requiring an office visit, they can add to the revenue stream and make care recommendations with no overhead — often at a cost savings for patients. IoT devices can gather all sorts of valuable patient data and are becoming more sophisticated and ubiquitous.Wearable health monitoring devices also exist for a variety of different types of patient demographics for routine or emergency use, including pregnant women, elderly patients and children with unique medical needs. Additionally, recreational wearables are increasingly common in households, as they can provide valuable information about the wearer’s vital or fitness statistics.However, the proliferation of these devices, not to mention the wide variety of environments in which patients will use them, create a significant challenge for healthcare IT departments. Some patients live in areas with slow cloud upload/download speeds; others might live or work in areas with extreme weather conditions, during which the device must continue to work.

\indent

Artificial intelligence. AI has become a disruptive innovation in healthcare that offers tremendous opportunities for better patient care and ROI.AI is playing an increasing role in triage and diagnoses, such as through medial image reading. While humans make the final call on diagnosis and treatment, AI models can gather and process large amounts of image data to offer a second opinion of sorts, which can be especially useful for hard-to-read scans.While this type of usage is in vogue, AI has ample opportunity to be a disruptive innovation in healthcare logistics and patient experience too.





\end{document}
