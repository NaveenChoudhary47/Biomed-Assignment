\documentclass[12pt]{article}
\usepackage[english]{babel}
\usepackage{natbib}
\usepackage{url}
\usepackage[utf8x]{inputenc}
\usepackage{amsmath}
\usepackage{graphicx}
\graphicspath{{images/}}
\usepackage{parskip}
\usepackage{fancyhdr}
\usepackage{vmargin}
\setmarginsrb{3 cm}{2.5 cm}{3 cm}{2.5 cm}{1 cm}{1.5 cm}{1 cm}{1.5 cm}

\title{Essay on 5 Medical Devices}								% Title
\author{21111034}								% Author
\date{25 Jan 2022}											% Date

\makeatletter
\let\thetitle\@title
\let\theauthor\@author
\let\thedate\@date
\makeatother

\pagestyle{fancy}
\fancyhf{}
\rhead{\theauthor}
\lhead{\thetitle}
\cfoot{\thepage}

\begin{document}

%%%%%%%%%%%%%%%%%%%%%%%%%%%%%%%%%%%%%%%%%%%%%%%%%%%%%%%%%%%%%%%%%%%%%%%%%%%%%%%%%%%%%%%%%

\begin{titlepage}
	\centering
    \vspace*{0.5 cm}
    \includegraphics[scale = 0.20]{logo.jpeg}\\[1.0 cm]	% University Logo
    \textsc{\LARGE  National Institute of Technology\newline\newline Raipur}\\[2.0 cm]	% University Name
	\textsc{\Large assignment 01}\\[0.5 cm]				% Course Code
	\rule{\linewidth}{0.2 mm} \\[0.4 cm]
	{ \huge \bfseries \thetitle}\\
	\rule{\linewidth}{0.2 mm} \\[1.5 cm]
	
	\begin{minipage}{0.4\textwidth}
		\begin{flushleft} \large
			\emph{Submitted To:}\\
			Saurabh Gupta\\
            Asst. Professor\\
            Department of Biomedical Engineering\\
			\end{flushleft}
			\end{minipage}~
			\begin{minipage}{0.4\textwidth}
            
			\begin{flushright} \large
			\emph{Submitted By :} \\
			Naveen Choudhary\\
            21111034\\
        First Semester\\
        Biomedical Engineering\\
		\end{flushright}
        
	\end{minipage}\\[2 cm]
	
	
    
    
    
    
	
\end{titlepage}



\section{BONE STIMULATOR}

The human bone grows in the process called OSTEOGENESIS, literally 'bone begining'. From birth until late adolscence or early adulthood, our bones elongate until we reach our maximum heights. However, the process of bone repair and regeneration will continue unti we die. Most damaged, broken or surgically altered bones will  heal themselves well, given sufficient time, blood supply and stabalization(like a plaster cast, air boot, pins, plates and screw for bone factures or the rods, screws and plates used for spinal fusions). 

\indent

Approximately 5-10 percent of individuals who have orthopaedic surgery or fractures will have bones that do not heal on their own or will be delayed in healing.  This is called non-union and it is a complication that is dealt with by an Orthopaedic specialist.  Non-union is diagnosed by taking x-rays of the area of the surgery or fracture over a period of weeks or months.  The x-ray images will show if there is new bone bridging the fractured or surgical area. If no new bone exists or just a small amount has formed during the time in which total healing should have occurred, non or delayed-union should be diagnosed. The patient may also have pain in the area that persists long after the initial fracture or surgery.

\indent

Non-union is a complication which causes chronic pain, dysfunction and hampers a person’s return to work and normal activities.  The most common reasons for non-union are: 1) the patient smokes; 2) they have multi-level spine fusion surgery; 3) other failed orthopaedic surgeries; 4) medical conditions, such as alcoholism, obesity or diabetes that can stunt bone healing; 5) the location of the bone in the body sometimes retards healing due to poor blood supply; or 6) infection.  Sometimes an otherwise healthy person will have non-union and we won’t know the reason, like Peyton Manning, the famous NFL Quarterback. That is called idiopathic non-union.

\indent

The best way to deal with non-union is to prevent it, whenever possible. If a patient is scheduled for surgery, they should quit smoking and address whatever issues may prevent healing, before their surgery.  If all factors are out of the patient’s control, they must be educated about the risks of non-union before proceeding with surgery.

\indent

There are a variety of ways to deal with non-union, some of which include surgery, internal or external fixation, bone grafting or the use of biologic bone substitute. The least invasive and only non-surgical option is a durable medical equipment device called a Bone Growth Stimulator. Bones create a mild electric field when they are healing or growing.  A bone growth stimulator (BGS) sends more energy to the healing bone surface through either pulsed electromagnetic or ultrasound waves, which helps the bone heal more quickly. The Orthopaedic physicians at OSC only use external bone growth stimulators.  Internal BGS units are expensive, implanted at the time of surgery and require a second surgery to remove the battery and wires of the unit from the body.

\indent

External BGS are available for almost any area of the body where a fracture or surgery has occurred.  They are available as either units that connect to electrodes which deliver current to the skin over the non-healing bone or as a type of brace or belt that can either be worn or upon which a patient can recline when sitting.  Typically, the BGS unit needs to be used for several hours a day over a three to nine month period to be effective and patient compliance is needed to ensure the best outcome.  The units are relatively comfortable, lightweight and safe.  They do not need adjustment, like TENs units.  The electric current is imperceptible and causes no discomfort to the patient.

\indent

This technology is expensive and costs range from $500-$5000, depending on the manufacturer and area of the body to be treated.  When ordered by an orthopaedic physician, BGS units are typically covered by most insurance carriers, but not all.  You may be covered for the entire cost of the unit or there may be some out of pocket expense.  Check with your insurance carrier if your physician suggests a Bone Growth Stimulator is right for you.

\newpage

\section{C-ARM}

A C-arm is an imaging scanner intensifier. The name derives from the C-shaped arm used to connect the x-ray source and x-ray detector to one another. C-arms have radiographic capabilities, though they are used primarily for fluoroscopic intraoperative imaging during surgical, orthopedic and emergency care procedures. The devices provide high-resolution X-ray images in real time, thus allowing the physician to monitor progress and immediately make any corrections.

\indent

An x-ray image intensifier (XRII) is an image intensifier that converts x-rays into visible light at higher intensity than mere fluorescent screens do. X-ray imaging systems use such intensifiers  (like fluoroscopes) to allow converting low-intensity x-rays to a conveniently bright visible light output.

\indent

Through its intensifying effect, the viewer can more easily see the structure of the imaged object than fluorescent screens alone. The XRII requires lower absorbed doses due to more efficient conversion of x-ray quanta to visible light.

\indent

There are two main configurations of permanently installed fluoroscopic systems. One class commonly utilizes a radiolucent patient examination table with an under-table mounted tube and an imaging system mounted over the table. The other is commonly referred to as a C-arm system that is used where greater flexibility in the examination process is needed.

\indent

The C-arm systems are commonly used for studies requiring the maximum positional flexibility such as:Angiography studies (peripheral, central and cerebral);
Therapeutic studies (Line placements, transjugular biopsies, TIPS stent, embolizations);
Cardiac studies;
Orthopedic procedures

\indent

Mobile Fluoroscopic System, also known as portable or mobile C-arm, comprises a generator (X-ray source) and an image intensifier or flat-panel detector. The C-shaped connecting element allows movement horizontally, vertically and around the swivel axes, so that X-ray images of the patient are produced from almost any angle.The generator emits X-rays that penetrate the patient’s body. The image intensifier or detector converts the X-rays into a visible image displayed on the C-arm monitor. Physician can check anatomical details such as bones and the position of implants and instruments at any time.

\indent

Flat-panel detectors (FDP) are increasingly replacing image intensifiers (II) on mobile C-arm systems, part of a migration of technology once available only in fixed room systems.The advantages of this technology include: lower patient dose and increased image quality and no deterioration of the image quality over time.
Despite FPD’s higher cost, the noteworthy changes in the physical size and accessibility for the patients is worth it.

\indent

Three-dimensional (3D) C-arm computed tomography is a new and innovative imaging technique. It uses two-dimensional (2D) X-ray projections acquired with a FDP C-arm system to generate CT-like images. To this end, the C-arm system performs a sweep around the patient, acquiring up to several hundred 2D views. They serve as input for 3D cone-beam reconstruction. Resulting voxel data sets can be visualized either as cross-sectional images or as 3D data sets using different volume rendering techniques. Initially targeted at 3D high-contrast neurovascular applications, 3D C-arm imaging has continuously improved over the years and  now provides CT-like soft-tissue image quality. In combination with 2D fluoroscopic or radiographic imaging, 3D C-arm imaging provides valuable information for therapy planning, guidance, and outcome assessment all in the interventional suite.

\indent

Failure of the x-ray beam collimation may lead to primary beam x-ray exposure outside of the selected image intensifier input area. This would result in image degradation. Light generated outside the area of the image intensifier input at magnification causes additional loss of contrast of the image with increased noise. Additionally, unnecessary additional dose to the patient would result. If the C-arm or fittings are damaged, the x-ray tube and intensifier may become misaligned resulting in image degradation or loss, as well as presenting a potential injury to staff and patient if the structural integrity of the C-arm or mounted components are compromised.

\newpage

\section{LAB INCUBATOR}

Incubator, in microbiology, is an insulated and enclosed device that provides an optimal condition of temperature, humidity, and other environmental conditions required for the growth of organisms.An incubator is a piece of vital laboratory equipment necessary for the cultivation of microorganisms under artificial conditions.An incubator can be used for the cultivation of both unicellular and multicellular organisms.

\indent

A microbial incubator is made up of various units, some of which are:\\
Cabinet\\
The cabinet is the main body of the incubator consisting of the double-walled cuboidal enclosure with a capacity ranging from 20 to 800L.
Door\\
A door is present in all incubators to close the insulated cabinet.The door also has insulation of its own. 
Control Panel\\
On the outer wall of the incubator is a control panel with all the switches and indicators that allows the parameters of the incubator to be controlled.
Thermostat\\
A thermostat is used to set the desired temperature of the incubator.\
Perforated shelves\\
Bound to the inner wall are some perforated shelves onto which the plates with the culture media are placed.
Asbestos door gasket\\
The asbestos door gasket provides an almost airtight seal between the door and the cabinet.
L-shaped thermometer\\
A thermometer is placed on the top part of the outer wall of the incubator.
HEPA filters\\
Some advanced incubators are also provided with HEPA filters to lower the possible contamination created due to airflow.

\indent

An incubator is based on the principle that microorganisms require a particular set of parameters for their growth and development.In an incubator, the thermostat maintains a constant temperature that can be read from the outside via the thermometer.In an incubator, the thermostat maintains a constant temperature that can be read from the outside via the thermometer.During the heating cycle, the thermostat heats the incubator, and during the no-heating period, the heating is stopped, and the incubator is cooled by radiating heat to the surrounding.Insulation from the outside creates an isolated condition inside the cabinet, which allows the microbes to grow effectively.Similarly, other parameters like humidity and airflow are also maintained through different mechanisms that create an environment similar to the natural environment of the organisms.Similarly, they are provided with adjustments for maintaining the concentration of CO2 to balance the pH and humidity required for the growth of the organisms.Variation of the incubator like a shaking incubator is also available, which allows for the continuous movement of the culture required for cell aeration and solubility studies.

\indent

Once the cultures of organisms are created, the culture plates are to be placed inside an incubator at the desired temperature and required period of time. In most clinical laboratories, the usual temperature to be maintained is 35–37°C for bacteria.The door of the incubator is then kept closed, and the incubator is switched on. The incubator has to be heated up to the desired temperature of the growth of the particular organism. The thermometer can be used to see if the temprature has reached.In the meantime, if the organism requires a particular concentration of CO2 or a specific humidity, those parameters should also be set in the incubator.

\indent

On the basis of the presence of a particular parameter or the purpose of the incubator, incubators are divided into the following types:\\
Benchtop incubators\\
These incubators are the basic types of incubators with temperature control and insulation.\\
CO2 incubators\\
CO2 incubators are the special kinds of incubators that are provided with automatic control of CO2 and humidity.This type of incubator is used for the growth of the cultivation of different bacteria requiring 5-10 percent of CO2 concentration.\\
Cooled incubators\\
These incubators are use for incubation at temperatures below the ambient, incubators are fitted with modified refrigeration systems with heating and cooling controls.\\
Shaker incubator\\
A thermostatically controlled shaker incubator is another piece of apparatus used to cultivate microorganisms.Its advantage is that it provides a rapid and uniform transfer of heat to the culture vessel, and its agitation provides increased aeration, resulting in acceleration of growth.

\indent

Incubators have a wide range of applications in various areas including cell culture, pharmaceutical studies, hematological studies, and biochemical studies.
Incubators are used to grow microbial culture or cell cultures.They can also be used to maintain the culture of organisms to be used later.Some incubators are used to increase the growth rate of organisms, having a prolonged growth rate in the natural environment.Specific incubators are used for the reproduction of microbial colonies and subsequent determination of biochemical oxygen demand.These are also used for breeding of insects and hatching of eggs in zoology.Incubators also provide a controlled condition for sample storage before they can be processed in the laboratories.

\newpage

\section{NEEDLE FREE INJECTOR}

Needle free injection technology (NFIT)is an extremely broad concept which include a wide range of drug delivery systems that drive drugs through the skin using any of the forces as Lorentz, Shock waves, pressure by gas or electrophoresis which propels the drug through the skin, virtually nullifying the use of hypodermic needle. This technology is not only touted to be beneficial for the pharma industry but developing world too find it highly useful in mass immunization programmes, bypassing the chances of needle stick injuries and avoiding other complications including those arising due to multiple use of single needle. The NFIT devices can be classified based on their working, type of load, mechanism of drug delivery and site of delivery. To administer a stable, safe and an effective dose through NFIT, the sterility, shelf life and viscosity of drug are the main components which should be taken care of. Technically superior needle-free injection systems are able to administer highly viscous drug products which cannot be administered by traditional needle and syringe systems, further adding to the usefulness of the technology. NFIT devices can be manufactured in a variety of ways; however the widely employed procedure to manufacture it is by injection molding technique. There are many variants of this technology which are being marketed, such as Bioject® ZetaJetTM, Vitajet 3, Tev-Tropin® and so on. Larger investment has been made in developing this technology with several devices already being available in the market post FDA clearance and a great market worldwide.

\indent

Needle free injection technology (NFIT) encompasses a wide range of drug delivery systems that drive drugs through the skin using any of the forces as Lorentz, shock waves, pressure by gas or electrophoresis which propels the drug through the skin, virtually nullifying the use of hypodermic needle.The devices as such are available in reusable forms. In contrast to the traditional syringes, NFIT not only gives the user freedom from unnecessary pain but drugs in the form of solid pallets can also be administered. The future of this technology is promising ensuring virtually painless and highly efficient drug delivery. The major drawback associated with this technology is postadministration “wetness” of the skin which may, if not taken care of, harbor dust and other untoward impurities.This technology is being backed by organizations as World Health Organization, Centers for Disease Control and Prevention and various groups including Bill and Melinda Gates Foundation. This technology is not only touted to be beneficial for the pharma industry but developing world too find it highly useful in mass immunization programs, bypassing the chances of needle stick injuries and avoiding other complications including those arising due to multiple uses of single needle.Better patient compliance has been observed.

\indent

Syringes and hypodermic needles have been used to administer the drug to the body for more than 150 years. It was in 1844; hollow needles were devized and the first injection was administered soon. However, only those drugs could be given which possessed a specific combination of physiochemical properties.Primitive syringes were one-piece metal systems attached to rubber plunger used to inject the drug. These syringes were reused and were difficult to sterilize. The evolution of modern day syringe systems has led to the involvement of medical-grade stainless-steel as hypodermic needles while the body is made of plastic, developing the syringes as a two-part disposable system.However, the technical advancements and bioengineering capabilities have led to the emergence of various “newer” active enhancements, designed so as to circumvent the barrier function of the stratum corneum.

\indent

NFIT are novel ways of direct transfer of medicine through the skin, without breaching the integrity of the skin or even piercing it. These devices can be used to drive medicaments into the muscle too.NFIT has shown promising results in mass immunization and vaccination programs. These systems are virtually painless as they avoid the use of conventional needles.

\indent

NFIT harnesses energy stronger enough to propel a premeasured dose of a particular drug formulation, loaded in specific unique “cassettes” which can be rigged with the system.These forces may be generated from any of the ways ranging from high-pressure fluids including gases, electro-magnetic forces, shock waves or any form of energy capable enough to impart motion to the medicament.

\indent

Needle free technology are capable of delivering a wide spectrum of medicinal formulations into the body with the same bioequivalence as that which could have been achieved by drug administration by a two-piece syringe system, without inflating unnecessary pain to the patients. These devices are very easy to be used, don't require any expert supervision or handling, easy to store, and dispose.

\indent

These devices are suitable for delivery of drugs to some of the most sensitive parts of the body like cornea. They are efficient to administer intra-muscular, subcutaneous and intra-dermal injections. These systems require a power source which may be obtained either physically or by the application of some force. The drug is forced and is ejected through a superfine nozzle at speeds near about to that of sound.

\newpage

\section{EXOSKELETON}

Say the term ‘power suit’ and most people think of bold corporate attire. But the expression takes on new meaning when it refers to a powered “exoskeleton,” like Ellen Ripley’s power loader in "Aliens," or Iron Man’s armor from the Marvel films and comic books.Until a few years ago, such exoskeletons — metal frameworks fitted with motorized "muscles" that can multiply the wearers’ strength far beyond that of normal humans — were entirely fictional. The only real-world exoskeletons were the natural external coverings of animals such as beetles and crabs; protective outer structures that provide a stiff frame upon which their muscles can push against to move their bodies around.

\indent

Today, powered exoskeleton suits are becoming a reality; perhaps several hundred commercial and experimental exosuits now operate globally. But until somebody invents the equivalent of the palm-sized power plant Tony Stark wears in his chest, real-world exoskeleton suits will have to make do with an all-too-limited power supply — and much less spectacular capabilities — unless they can stay tethered to electrical cables in factories or at worksites. But any need for power cables runs counter to technology’s fundamental long-term aim: enhanced individual mobility for anyone, anywhere.

\indent

Even if current exoskeleton suits don’t bestow superhuman powers, they will increasingly impact people’s lives by helping the infirm get around safely and independently, and assisting workers to perform hazardous jobs with fewer injuries, no small task for anyone — superhero and average Joe included. Add the fact that a new generation of ‘soft’ exoskeleton suits is seemingly on the way and in time nearly everybody should be enabled to wear a power suit.

\indent

One of the few people on the planet who has spent considerable time wearing a powered exoskeleton is Steven Sanchez, Chief Pilot for suitX. He test-drives the Berkeley, Calif.-based start-up’s Phoenix, a motorized, lower-body framework that rather amazingly enables people with severe mobility issues to rise up from their wheelchairs and walk.Sanchez, a paraplegic since a BMX bike accident in 2004, is excited about the future of robotic exoskeletons. “This technology can give you the kind of mobility you want,” he says. “It physically opens up all kinds of access that’s been lacking, which makes for greater independence and a better quality of life.”But just as important as the mental side is the social benefit, Sanchez continues. “It lets you break out of the ‘wheel-chair bubble’” which often defines the interactions that disabled individuals have with others. “Just being able to walk up to people really changes how they react to you,” he says.

\indent

Few people have tried exoskeletons because they're still very costly. Even though relatively few exosuits exist, the list of potential users is long and growing. Beyond the many disabled and elderly people who could make use of the technology, so too could their nurses and caregivers, with health insurance covering the bill. In addition, shipping and industrial workers, loggers and miners would benefit from job-assist exosuits and robotic wearables, perhaps with employers writing off the costs as a safety device. Meanwhile, governments could soon start outfitting fire-fighters, EMS, and disaster personnel, combat troops and logistics specialists with protective exoskeletons.

\indent

Cheaper, more capable exoskeletons are on the way, according to Dan Kara, veteran robotics industry analyst at ABI Research. “An entire wearable robotics industry, today comprising around forty R and D groups worldwide, is coalescing that should become a 2-billion dollar global market by 2025.”“Lower body exoskeletons for rehabilitation or as quality-of-life enablers are available today,” he says, but systems that augment or amplify on-the-job performance are next. “This means assisting with industrial tasks that require heavy lifting, or easing extended standing, squatting, bending, or walking in manufacturing facilities and in the construction and farming industries.” Overexertion by workers costs employers 15 billion dollar a year in compensation, so providing protective exosuits could make good business sense.


\indent

Although limited power supply is still a huge constraint, today's batteries may prove to be enough as exosuit tech morphs into new forms. Not only are passive exoskeletons arriving on the wearable robotics catwalk, but many new exoskeletons have gone soft, eschewing metal frames altogether for flexible fabric and artificial muscles. The new trend stems from a DARPA-funded program dubbed Warrior Web, which seeks to prevent damage to injury-prone areas of the body and minimize fatigue.

\indent

Extrapolation is perilous, but exosuits could one day become mass-produced personal mobility machines for everyone: a fashionable, high-tech device that enables independence, improves overall physical well-being, and operates as a status symbol.

\end{document}
